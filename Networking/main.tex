\documentclass[11pt,a4paper]{article}

% Essential packages
\usepackage[utf8]{inputenc}
\usepackage[T1]{fontenc}
\usepackage{lmodern}
\usepackage[margin=1in]{geometry}
\usepackage{xcolor}
\usepackage{colortbl}
\usepackage{hyperref}
\usepackage{graphicx}
\usepackage{enumitem}
\usepackage{booktabs}
\usepackage{listings}
\usepackage[breakable,skins,theorems]{tcolorbox}
\usepackage{titlesec}
\usepackage{fancyhdr}
\usepackage{multirow}
\usepackage{wrapfig}
\usepackage{microtype}

% Custom colors
\definecolor{codebackground}{rgb}{0.95,0.95,0.95}
\definecolor{commandcolor}{rgb}{0.8,0.0,0.0}
\definecolor{outputcolor}{rgb}{0.0,0.0,0.6}
\definecolor{commentcolor}{rgb}{0.0,0.5,0.0}
\definecolor{sectioncolor}{rgb}{0.0,0.3,0.5}
\definecolor{subsectioncolor}{rgb}{0.0,0.4,0.4}
\definecolor{warningcolor}{rgb}{0.8,0.3,0.0}
\definecolor{tiphighlight}{rgb}{0.95,0.95,0.7}

% Custom colors - adding Kali-like terminal colors
\definecolor{kalibackground}{rgb}{0.15,0.15,0.15}
\definecolor{kalitext}{rgb}{0.4,0.7,1.0}  % Blue for all text
\definecolor{kaliprompt}{rgb}{0.2,0.8,0.8}  % Cyan for prompt
\definecolor{kalicommand}{rgb}{0.4,0.7,1.0}  % Blue for commands
\definecolor{kalioutput}{rgb}{0.4,0.7,1.0}  % Blue for output
\definecolor{kaliurl}{rgb}{0.4,0.7,1.0}  % Blue for URLs
\definecolor{kaliheader}{rgb}{0.4,0.7,1.0}  % Blue for headers

% Style settings
\hypersetup{
    colorlinks=true,
    linkcolor=sectioncolor,
    urlcolor=blue,
    citecolor=green
}

% Header & footer
\pagestyle{fancy}
\fancyhf{}
\rhead{Cybersecurity Notes}
\lhead{Nmap Cheatsheet}
\rfoot{Page \thepage}
\lfoot{\today}

% Title formatting
\titleformat{\section}
{\color{sectioncolor}\normalfont\Large\bfseries}
{\color{sectioncolor}\thesection}{1em}{}

\titleformat{\subsection}
{\color{subsectioncolor}\normalfont\large\bfseries}
{\color{subsectioncolor}\thesubsection}{1em}{}

% Custom environments
\newenvironment{commandbox}[1][]{
    \begin{tcolorbox}[
        colback=kalibackground,
        colframe=commandcolor,
        fonttitle=\bfseries\color{white},
        title=#1,
        breakable=true
    ]
}{
    \end{tcolorbox}
}

\newenvironment{tipbox}[1][Tip]{
    \begin{tcolorbox}[
        colback=tiphighlight,
        colframe=warningcolor,
        fonttitle=\bfseries\color{white},
        title=#1
    ]
}{
    \end{tcolorbox}
}

% Code listing settings
\lstset{
    backgroundcolor=\color{kalibackground},
    basicstyle=\footnotesize\ttfamily\color{warningcolor},
    breakatwhitespace=false,
    breaklines=true,
    captionpos=b,
    commentstyle=\color{kalitext},
    keepspaces=true,
    keywordstyle=\color{kalitext},
    showspaces=false,
    showstringspaces=false,
    showtabs=false,
    tabsize=2,
    moredelim=[il][\color{commentcolor}]{\$\ },
    stringstyle=\color{kalitext},
    moredelim=[is][\color{kalitext}]{§}{§}
}

\begin{document}

\begin{center}
    \begin{tcolorbox}[width=\textwidth, colback=sectioncolor!20, colframe=sectioncolor]
        \centering
        {\Huge \textbf{Nmap Cheatsheet}}\\[0.5em]
    \end{tcolorbox}
\end{center}

\tableofcontents
\clearpage

\vspace{1em}
\begin{tcolorbox}[colback=codebackground, colframe=warningcolor]
    This document shall contain everything I have learned so far about nmap. I must keep it up-to-date because I feel a bit lost in this cybersecurity journey to be frank.
\end{tcolorbox}

\section{Basic Nmap Commands}

\subsection{Service Version Detection (-sV)}

This is the Service Detection flag \textit{(yes; \texttt{-sV} is a single flag, not a combination of both \texttt{s} AND \texttt{V})}, which will tell you the name and description of the identified services.
\begin{commandbox}[Service Version Detection]
\begin{lstlisting}[language=bash]
$ sudo nmap -sV {target_IP}
\end{lstlisting}

\begin{lstlisting}[basicstyle=\ttfamily\small]
§Starting Nmap 7.80 ( https://nmap.org ) at 2025-03-28 06:27 GMT
Nmap scan report for 10.10.171.202
Host is up (0.00011s latency).
Not shown: 994 closed ports
PORT     STATE SERVICE    VERSION
7/tcp    open  echo
9/tcp    open  tcpwrapped
13/tcp   open  daytime?
17/tcp   open  qotd?
22/tcp   open  ssh        OpenSSH 9.6p1 Ubuntu 3ubuntu13.5 (Ubuntu Linux; protocol 2.0)
8008/tcp open  http       lighttpd 1.4.74

2 services unrecognized despite returning data. If you know the service/version, please submit the following fingerprints at https://nmap.org/cgi-bin/submit.cgi?new-service :
[Detailed fingerprint data omitted for brevity]

MAC Address: 02:89:EC:B7:74:EF (Unknown)
Service Info: OS: Linux; CPE: cpe:/o:linux:linux_kernel

Service detection performed. Please report any incorrect results at https://nmap.org/submit/ .
Nmap done: 1 IP address (1 host up) scanned in 14.00 seconds§
\end{lstlisting}
\end{commandbox}

\end{document}
